All five to ten measurements for each task can be found in the appendix.

\subsection{Startup Phase}
\begin{table}[h]
    \centering
    \resizebox{\columnwidth}{!}{%
        \input{sections/tables/Blatt 1-1\) Booting_average.tex}
    }
    \caption{The average mAs for the Startup phase.}
    \label{tab_boot_avg}
\end{table}

\verb!startMeasurment()! is called after the initialization of the pins and
\verb!stopMeasurment()! is called when every internet functionality is initialized.
To restart the measurement, we disconnected the power supply of the ESP.
The normal connecting time to Wifi was 5 seconds. In the university, we would measure
different mAs because it takes longer to connect. DFS(80,240) showed the best performance
(see table \ref{tab_boot_avg}) because
it might have faster calculation than 80MHz due to dynamic frequency scaling.

\subsection{Wifi Savings}
To indicate Internet functionalities were initialized correctly, we give a semaphore.
After the semaphore can be taken, the ESP starts the measurement
and waits 10 seconds in a \verb!vTaskDelay()!. We can see that the bigger the beacon
interval is the smaller the power consumption gets (see table \ref{tab_wifi_avg}).
However, I did not test the drawbacks of a higher beacon interval.
In addition, here we implemented an automatic restart with \verb!esp_restart()! after
the measurements.

\begin{table}[h]
    \centering
    \resizebox{\columnwidth}{!}{%
        \input{sections/tables/Blatt 2-2\) Wifi_average.tex}
    }
    \caption{The average mAs in Wifi savings for a corresponding beacon interval of 1, 2, 4, 16, 128, and 256.}
    \label{tab_wifi_avg}
\end{table}


\subsection{Calculation Phase}

Internet- and display functionalities were switched off by a \verb!#define!. In addition,
we had to insert a \verb!vTaskDelay(10)! in the calculation \verb!for!-loop.
Because the variance of the temporal execution and power consumption was not high, we
only performed five measurements instead of ten. However, we see that again DFS(80,240) + ALS
performs best. Even if 240MHz was a bit faster, it had a higher energy consumption.
We point out, that even if the slower frequency had a slower execution, the power consumption
of all frequencies are similar(as we see in table \ref{tab_cal_avg}).

\begin{table}[h]
    \centering
    \resizebox{\columnwidth}{!}{%
        \input{sections/tables/Blatt 3-3\) Calculation_average.tex}
    }
    \caption{The average mAs and seconds during the calculation phase. Seconds are marked with the column name ``Time''.}
    \label{tab_cal_avg}
\end{table}



\subsection{Parallelization of Calculation Phase}
One of the most effective improvements of the calculation code is to split equally the
most outer loop to the two cores. Such that core 1 executes the $t$ from 0 to 7 and core 2
executes $t$ from 8 to 15. That reduces the execution time by a factor of $0.5$
(see table \ref{tab_par_cal_avg}). Of course, there are no huge energy savings because
we are using two cores at the same time.

\begin{table}[h]
    \centering
    \resizebox{\columnwidth}{!}{%
        \begin{tabular}{lrlrlr}
    \toprule
    80 MHz & \specialcell{Time                           \\ 80 MHz} & 160 MHz &  \specialcell{Time \\ 160 MHz}  & 240 MHz &  \specialcell{Time \\ 240 MHz} \\
    \midrule
    986,4  & 28                & 816,4 & 19 & 758,2 & 16 \\
    \bottomrule
\end{tabular}

    }
    \caption{The average mAs and seconds during the paralleled calculation phase. Seconds are marked with the column name ``Time''.}
    \label{tab_par_cal_avg}
\end{table}


\subsection{Light and Deep Sleep}
We use DFS(80,240) + ALS with Wifi savings and a beacon interval of 128. So, before entering
for 10 seconds each sleep mode, internet will be activated. \\
We see that there is a higher energy consumption with activated display and deep sleep
consumed way less energy than light sleep, see table \ref{tab_sleep_avg}.
Deep sleep consumed $0.1$ mAs for the 10 seconds without the display and $1$ mAs
with the display.
There is that inaccuracy in the table \ref{tab_sleep_avg} because the measurement is stopped
after booting the ESP. So, it reveals how much energy is used to wake up form deep sleep too.


\begin{table}[h]
    \centering
    \resizebox{\columnwidth}{!}{%
        \begin{tabular}{llrr}
\toprule
Light no display & Light with display &  Deep no display &  Deep with display \\
\midrule
            30,2 &               51,6 &                3 &                  5 \\
\bottomrule
\end{tabular}

    }
    \caption{The average mAs for the sleep modes.}
    \label{tab_sleep_avg}
\end{table}


\subsection{Code explanations}
The overall idea is to stay in deep sleep as long as possible and add there events to the buffer.
We never tested how long the ESP might stay in deep sleep,
such that it wakes up after \verb!WAKEUP_AFTER! seconds. Currently, it is limited to 20 minutes.
Before sleeping 20 minutes after the first boot, it initializes internet parallel to
activity receiving new events. This is necessary, first, because the last implementation
rebooted if after 10 tries of established Wifi connection, that lead to a deletion of the buffer.
This version tries infinitely to connect to Wifi.
Second, the Wifi connection might take some minutes. During that time we would not detect
incoming new events if the internet initialization is not parallel. So, waking up if the buffer is full
is not an option because then this functionality is broken.
After this, it will wait for the semaphore \verb!xInternetActive!
before analyzing the buffer. The analyzing algorithm is the same as in the last report; however,
this time we do not use the NVS. If the algorithm detected an increase or decrease in the count,
it stores that in a \verb!cJSON! object which is hold in the RAM.
After emptying the buffer, the \verb!cJSON! object is converted to a string, and it is sent via MQTT to
the database. The last step before entering deep sleep is to check whether it is 23 o'clock.
Then, we reset the count and enter the deep sleep for 7 hours. If that is not the case,
we enter deep sleep for \verb!WAKEUP_AFTER! seconds.\\
There is the possibility to refresh the ESP's display with a button because it can occur that the display's values
are not up-to-date. Under the roof, this button forces the ESP to wake up from deep sleep and
performs then the above described procedure. \\
We renamed the \verb!wakeup_stub()! to \verb!wakeup_routine()!.





\subsection{Battery lifetime}
We are estimating the battery lifetime for our implementation that includes analyzing the buffer.
Because the ESP uses about $900 mAs$ to boot from deep sleep, then analyzing a full buffer takes about
1 seconds such that one cycle consumes about $1000 mAs = 16.6 mAh$. So, the ESP would
live $96$ cycles with a $1600 mAh$ battery. Currently, one cycle is $20$ minutes long, such that the
lifetime would be $(20*96)/24 = 32 h$. \\
A longer lifetime can be archived by increasing the buffer to $400$
events and increasing the cycle to $3 h$. This setup will avoid buffer overflows if we assume that
there are four events for an in- or out-going event, and 30 persons entering or leaving the room
approximately every 2 hours, such that the buffer has to handle $240$ events every three hours.
This would result in a lifetime of $(3*96)/24 = 12$ days. If we operate more risky and
empty the buffer every 4 hours we archive a lifetime of $16$ days.\\
This is in theory the longest lifetime if we assume that an event needs $(8+32+32)/8$ bytes:
8 bits for the sensor ID, 32 bits for the time and additionally 32 bits for managing the struct.
Further, we estimate that the accessible area of the 8 kB RTC RAM is 4000 bytes, which
results in a max buffer size of $4000/((8+32+32)/8) = 444$ events.
